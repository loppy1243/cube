$R$ is the rotation of the $x_3$-face; $r$ is the reflection in the $x_1x_3$-plane; and $s$ is
the rotation ("spin") that holds 4 and 7 fixed. It is easy to see, from visualization or
computation, that the orders of these elements are
\[ |\gen R| = 4,\quad |\gen{r_1}| = |\gen{r_2}| = 2,\quad |\gen s| = 3. \]
To show that these are indeed generators, it will be useful to consider the reflections
$r_{12}$ and $r_{23}$ about the respective $x_1x_2$-, $x_2x_3$-planes (noting that
$r=r_{13}$), as well as the rotations $R_1, R_2, R_3$ about the respective $x_1$-, $x_2$-,
$x_3$-axes (noting that $R = R_3^{-1}$). We see that
\begin{alignat*}3
srs^{-1} &= (36)(54)(18)(72) = r_{12},\quad & RrR^{-1}     &= (23)(41)(85)(67) = r_{23}, \\
sRs^{-1} &= (3654)(2781) = R_1,\quad        & s^{2}Rs^{-2} &= (5814)(6723) = R_2^{-1}.
  \numberthis\label{eqn:gen->sp}
\end{alignat*}
So these elements (and their inverses) are in $\gen{R,r,s}$. Consider now the $\R^3$
representations of $r_{12}, r_{23}, r_{13}, R_1, R_2, R_3$, which are easily obtainable by
considering the transformation of the faces of \Cube:
\begin{alignat*}3
 r_{12} &\mapsto \begin{pmatrix}\s1 &\s0 &\s0\s \\\s0 &\s1 &\s0\s \\\s0 &\s0 & -1\s \end{pmatrix},\quad
&r_{23} &\mapsto \begin{pmatrix} -1 &\s0 &\s0\s \\\s0 &\s1 &\s0\s \\\s0 &\s0 &\s1\s \end{pmatrix},\quad
&r_{13} &\mapsto \begin{pmatrix}\s1 &\s0 &\s0\s \\\s0 & -1 &\s0\s \\\s0 &\s0 &\s1\s \end{pmatrix}, \\
 R_1    &\mapsto \begin{pmatrix}\s1 &\s0 &\s0\s \\\s0 &\s0 &\s1\s \\\s0 & -1 &\s0\s \end{pmatrix},\quad
&R_2    &\mapsto \begin{pmatrix}\s0 &\s0 & -1\s \\\s1 &\s0 &\s0\s \\\s0 &\s1 &\s0\s \end{pmatrix},\quad
&R_3    &\mapsto \begin{pmatrix}\s0 &\s1 &\s0\s \\ -1 &\s0 &\s0\s \\\s0 &\s0 &\s1\s \end{pmatrix}.
\end{alignat*}
In particular, it is easiest to study the combinations
\begin{subequations}\label{eqn:sp-mat}
\begin{alignat*}6
 r_{12}    &\mapsto \begin{pmatrix}\s1 &\s0 &\s0\s \\\s0 &\s1 &\s0\s \\\s0 &\s0 & -1\s \end{pmatrix},\quad
&r_{23}    &\mapsto \begin{pmatrix} -1 &\s0 &\s0\s \\\s0 &\s1 &\s0\s \\\s0 &\s0 &\s1\s \end{pmatrix},\quad
&r_{13}    &\mapsto \begin{pmatrix}\s1 &\s0 &\s0\s \\\s0 & -1 &\s0\s \\\s0 &\s0 &\s1\s \end{pmatrix}, \numberthis\label{eqn:s-mat} \\
 r_{12}R_1 &\mapsto \begin{pmatrix}\s1 &\s0 &\s0\s \\\s0 &\s0 &\s1\s \\\s0 &\s1 &\s0\s \end{pmatrix},\quad
&r_{23}R_2 &\mapsto \begin{pmatrix}\s0 &\s0 &\s1\s \\\s1 &\s0 &\s0\s \\\s0 &\s1 &\s0\s \end{pmatrix},\quad
&rR_3      &\mapsto \begin{pmatrix}\s0 &\s1 &\s0\s \\\s1 &\s0 &\s0\s \\\s0 &\s0 &\s1\s \end{pmatrix}. \numberthis\label{eqn:p-mat}
\end{alignat*}
\end{subequations}
The matrices of $r_{12}R_1, r_{23}R_2, r_{13}R_3$ are precisely the standard permutation matrices
of $(23), (123), (12)$, respectively, and generate all six possible permutation matrices:
\[
(1),\quad (12),\quad (23),\quad
\underbrace{(13)}_{=(12)(23)},\quad
(123),\quad
\underbrace{(321)}_{=(12)(123)}. \]
It was shown in Part~\ref{part:construction} that all elements of $M \in \IsomC$ are of the
form (Eq.~\ref{eqn:isommat})
\[ M = (\pm\unitv x_a, \pm\unitv x_b, \pm\unitv x_c) \]
for some permutation $(a, b, c)$ of $(1, 2, 3)$ and choice of signs, and that each choice of
permutation and sign corresponds to a unique element of \IsomC. The elements
Eq.~\ref{eqn:s-mat} allow us to choose the sign, and the elements Eq.~\ref{eqn:p-mat} allow us
to choose the permutation. Given a choice of sign $\iota = (\iota_1, \iota_2, \iota_3)$ and
permutaton $\sigma \in S_3$, we can now build every element $M^\iota_\sigma \in \IsomC$ as
\begin{equation}\label{eqn:sp-rep}
M^\iota_\sigma = m(\sigma)r_{23}^{n(\iota_1)}r_{13}^{n(\iota_2)}r_{12}^{n(\iota_3)}
               = r_{23}^{n(\iota_{\sigma(1)})}r_{13}^{n(\iota_{\sigma(2)})}r_{12}^{n(\iota_{\sigma(3)})}m(\sigma),
\end{equation}
where
\begin{gather*}
n(x) = \frac{1-x}2, \\
\begin{aligned}
m((1))\n &= 1,\quad                  & m((12))\n &= r_{13}R_3,\quad            & m((23))\n &= r_{12}R_1,\quad \\
m((13))  &= r_{12}R_1r_{13}R_3,\quad & m((123))  &= r_{23}R_2,\quad & m((321)) &= r_{13}R_3r_{23}R_2.
\end{aligned}
\end{gather*}
It is thus that $\gen{R, r, s} = \IsomC$.

\subsection*{Aside}
The second equality in Eq.~\ref{eqn:sp-rep} comes from the fact that right multiplication by
e.g. $r_{23}$ sets the sign of the first \emph{column} of $m(\sigma)$, and left multiplication
sets the sign of the first \emph{row}. Each row and column of $m(\sigma)$ contains exactly one
1, and the map that links corresponding rows and columns is exactly $\sigma$.

It is clear that $\gen{r_{23}, r_{13}, r_{12}}$ and the set of permutation matrices $P_3 =
\set{m(\sigma)}[\sigma \in S_3] \iso S_3$ are trivially intersecting subgroups of \IsomC. This
second equality then shows us that $\gen{r_{23}, r_{13}, r_{12}}$ is a normal subgroup of
\IsomC, and it follows that
\[ \quot{\IsomC}{\gen{r_{23}, r_{13}, r_{12}}} \iso S_3. \]
Of course, this is not surprising since we've already shown that \IsomC\ is the set of signed
permutation matrices. Note however that this does not mean that \IsomC\ is a product group:
\[ \IsomC \not\iso C_2^3\times S_3, \]
where $C_2^3 \iso \gen{r_{23}, r_{13}, r_{12}}$ is the product of three order-two cyclic
groups. While Eq.~\ref{eqn:sp-rep} does tell us that $\gen{r_{23}, r_{13}, r_{12}}$ is normal,
it also explicitly shows that it does not commute with $P_3$.
