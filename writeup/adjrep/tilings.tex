We may generate interesting visual representations of the elements of \IsomC\ by considering
tilings of the adjacency representation matrices in the following way: for $I \in P_8$
corresponding to an elements of \IsomC, construct the matrices $A, B, C$ as
\[
A_{jk} = I_{j,(9-k)},\quad B_{jk} = I_{(9-j),k},\quad C_{jk} = I_{(9-j),(9-k)},\quad
j, k = 1,\dotsc,8,
\]
which are the reversals of the columns, rows, and columns and rows of $I$, respectively. Now
form the matrix $t$ and the infinite \emph{tiling matrix} $T(I)$ of $I$ by
\[
t = \begin{pmatrix}
I &\vline& A \\ \hline
B &\vline& C
\end{pmatrix},\quad
T_I = \begin{pmatrix}
\ddots&\vdots&\ddots&\vdots& \\
\cdots & t &\cdots& t & \cdots \\
\ddots&\vdots&\ddots&\vdots&\ddots \\
\cdots & t &\cdots& t & \cdots \\
&\vdots&\ddots&\vdots&\ddots
\end{pmatrix}.
\]
Portions of $T(I)$ are presented for each element of \IsomC\ in
Section~\ref{sec:tilings-T_F(C)} of the Appendix, with the generating matrix $I$ highlighted.

We can generalize this by realizing that for the matrix $F \in P_8$ where
\[
F = \begin{pmatrix}
0 & 0 & 0 & 0 & 0 & 0 & 0 & \one \\
0 & 0 & 0 & 0 & 0 & 0 & \one & 0 \\
0 & 0 & 0 & 0 & 0 & \one & 0 & 0 \\
0 & 0 & 0 & 0 & \one & 0 & 0 & 0 \\
0 & 0 & 0 & \one & 0 & 0 & 0 & 0 \\
0 & 0 & \one & 0 & 0 & 0 & 0 & 0 \\
0 & \one & 0 & 0 & 0 & 0 & 0 & 0 \\
\one & 0 & 0 & 0 & 0 & 0 & 0 & 0
\end{pmatrix},
\]
we have that
\[ A = I\trans F,\quad B = FI,\quad C = FI\trans F. \]
Since $F^2 = \mm 1$, we can view $T(I) = T_F(I)$ as successive conjugations of $I$ by $F$
along the $8\times8$ diagonal and counter-diagonal, along with the intermediate products $FI$
and $I\trans F$:
\[
T_F(I) = \begin{pmatrix}
\ddots  & \vdots        & \vdots  & \vdots   & \iddots \\
\cdots  &      FIF^{-1} & FI      &      FIF & \cdots  \\
\cdots  &       IF^{-1} & I       &       IF & \cdots  \\
\cdots  & F^{-1}IF^{-1} & F^{-1}I & F^{-1}IF & \cdots  \\
\iddots & \vdots        & \vdots  & \vdots   & \ddots
\end{pmatrix}
\]
Note that for this specific case $F = \trans F = F^{-1}$, but in general for any $K \in P_8$
we have $\trans K = K^{-1}$. $F$ is an isometry, specifically $I_{(87654321)}$, so there are
at most four isometries $\set{I, IF, FI, FIF}$ that generate the same tiling $T_F(I)$.

If we consider an arbitrary isometry $K \in P_8$ and generate $T_K(I)$, then if $K$ is of
order $a$, $T_K(I)$ repeats every $a$ columns and $a$ rows over from $I$. There are then at
most $a^2$ isometries that generate the same tiling based around $I$; this number is lowered
if some power of $K$ commutes with $I$. Most interesting is to note that the tiling
$T_K(\mm1)$, the tiling of the identity by $K$, is a pictorial representation of the cyclic
group $\gen K$ generated by $K$; we can easily identify the order of $K$ by noting where the
diagonal identity conjugation line repeats. By this method, I've found that the highest order
of an element of \IsomC\ is six.

$T_F(\IsomC)$ and $T_{\IsomC}(\mm 1)$ are provided in Sections~\ref{sec:tilings-T_F(C)}
and~\ref{sec:tilings-T_C(I)} in the Appendix. The tilings were generated by a Julia program I
wrote, which you can find online at \url{https://github.com/loppy1243/Cube}.
