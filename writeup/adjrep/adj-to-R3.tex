Recall the construction in Eq.~\ref{eqn:vert-to-face} of the standard basis vectors
$\set{\unitv x_j}_{j=1}^3$ from the vertex vectors $\set{\vv y_k}_{k=1}^8$:
\[
\unitv x_1 = \frac14(\vv y_1+\vv y_2+\vv y_7+\vv y_8),\quad
\unitv x_2 = \frac14(\vv y_1+\vv y_4+\vv y_5+\vv y_8),\quad
\unitv x_3 = \frac14(\vv y_1+\vv y_2+\vv y_3+\vv y_4).
\]
Define the columns of the matrix $Y \in \Mat\R3[8]$ as $\set{\vv y_k}_{k=1}^8$:
\[ Y_{ak} = [\vv y_k]_a,\quad a=1,2,3,\; k=1,\dotsc,8. \]
For an $M \in \IsomC$, since adjacency representation is faithful
(Eq.~\ref{eqn:adjrep-is-faithful}) there is a unique $I \in P_8$ corresponding to $M$. The matrix $M$ is understood most easily as a mapping of the faces of
\Cube; $I$ is understood as a mapping of the vertices. Multiplying by $Y$ by $I$ on the right
permutes the columns of $Y$ according to the permutation that $I$ represents. Constructing the
faces of \Cube\ from $YI$ then gives us the columns of $M$:
\begin{gather*}
M_{a,1} = \frac14\Bbrak(){[YI]_{a,1}+[YI]_{a,2}+[YI]_{a,7}+[YI]_{a,8}},\quad
M_{a,2} = \frac14\Bbrak(){[YI]_{a,1}+[YI]_{a,4}+[YI]_{a,5}+[YI]_{a,8}}, \\
M_{a,3} = \frac14\Bbrak(){[YI]_{a,1}+[YI]_{a,2}+[YI]_{a,3}+[YI]_{a,4}}.
\end{gather*}
We can write this in a nicer form by noting that by defintion $[YI]_{a,j} =
\sum_{k=1}^8Y_{ak}I_{kj}$, so
\begin{alignat*}3
[YI]_{a,1} + [YI]_{a,2} + [YI]_{a,7} + [YI]_{a,8}
 &= \sum_{k=1}^8Y_{ak}\brak(){I_{k,1} + I_{k,2} + I_{k,7} + I_{k,8}}
&&= [YJ(I)]_{a,1} \\
[YI]_{a,1} + [YI]_{a,4} + [YI]_{a,5} + [YI]_{a,8}
 &= \sum_{k=1}^8Y_{ak}\brak(){I_{k,1} + I_{k,4} + I_{k,5} + I_{k,8}}
&&= [YJ(I)]_{a,2} \\
[YI]_{a,1} + [YI]_{a,2} + [YI]_{a,3} + [YI]_{a,4}
 &= \sum_{k=1}^8Y_{ak}\brak(){I_{k,1} + I_{k,2} + I_{k,3} + I_{k,4}}
&&= [YJ(I)]_{a,3},
\end{alignat*}
where we've defined $J(I) \in \Mat\R8[3]$ as
\begin{gather*}
J_{k,1}(I) = I_{k,1} + I_{k,2} + I_{k,7} + I_{k,8},\quad
J_{k,2}(I) = I_{k,1} + I_{k,4} + I_{k,5} + I_{k,8}, \\
J_{k,3}(I) = I_{k,1} + I_{k,2} + I_{k,3} + I_{k,4}
\end{gather*}
for $k=1,\dotsc,8$. It follows that $M = \frac14YJ(I)$.

The inverse map is constructed simply by acting on the vertices $\set{y_k}_{k=1}^8$ with $M$,
which must permute them by some permutation $\sigma \in S_8$:
\[ My_k = y_{\sigma(k)},\quad k = 1,\dotsc,8, \]
The corresponding matrix in the adjacenecy representation is just $I_\sigma$ from
Eq.~\ref{eqn:8-permmat}.
